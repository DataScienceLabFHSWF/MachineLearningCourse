\documentclass{article}

% Language setting
% Replace `english' with e.g. `spanish' to change the document language
\usepackage[english]{babel}

% Set page size and margins
% Replace `letterpaper' with`a4paper' for UK/EU standard size
\usepackage[letterpaper,top=2cm,bottom=2cm,left=3cm,right=3cm,marginparwidth=1.75cm]{geometry}

% Useful packages
\usepackage{amsmath}
\usepackage{graphicx}
\usepackage[colorlinks=true, allcolors=blue]{hyperref}

\title{Machine Learning \\ Exercise 3: Usage of different regression algorithms in python}
\author{Prof. Dr. Thomas Kopinski}

\begin{document}
\maketitle

\begin{abstract}
The goal of this exercise is to use python and scikit-learn to perform logistic and polynomial regression on different datasets.
\end{abstract}

\section*{Task 1: Getting familiar with the regression algorithms}

\begin{itemize}
    \item Further information and examples about the implementation of logistic regression can be found in  \href{https://github.com/rasbt/machine-learning-book/blob/main/ch03/ch03.ipynb}{this} jupyter notebook.
    \item Additional information about polynomial regression in python can again be found \href{https://github.com/DataScienceLabFHSWF/machine-learning-book/blob/main/notebooks/ch09/ch09.ipynb}{here}.
    \item Make sure that you understand the concepts you are trying to implement before starting to write any code.
\end{itemize}

\section*{Task 2: Logistic Regression}

\begin{itemize}
   \item Download the MNIST Dataset with the scikit-learn function: \\ \emph{sklearn.datasets.fetch\_openml("mnist\_784")}
   \item Look into the data and analyze the data structure (eg. data and label shape)
   \item Why is logistic regression a good approach for this dataset? What is the difference when compared to linear regression?
   \item Plot some samples of the data with matplotlib
   \item Split the data into training and test dataset
   \item Classify the MNIST dataset with a logistic regression model (scikit-learn)
   \item Evaluate the model performance using the accuracy
   \item Utilize and interpret the output of the two scikit-learn methods \emph{proba()} and \emph{predict\_proba()}. What is the difference here?
   \item Display a sample of the wrong predictions to understand the models shortfalls
\end{itemize}

\section*{Task 3: Polynomial Regression}

\begin{itemize}
    \item What is the use-case for Polynomial Regression? Why and when would you use it instead of Linear Regression?
    \item Read the data \\
    x = np.arange(0, 30) \\
    y = [3, 4, 5, 7, 10, 8, 9, 10, 10, 23, 27, 44, 50, 63, 67, 60, 62, 70, 75, 88, 81, 87, 95, 100, 108, 135, 151, 160, 169, 179]
    \item Use matplotlib to display the data
    \item Create polynomial features of degree 2, 5 and 10 from the input data using the\\ \emph{sklearn.preprocessing.PolynomialFeatures()} function
    \item Use Linear Regression models to fit to the transformed and untransformed data. 
    \item Plot the models together with the data and evaluate the model performances. What do you notice? What might be the problem here when using higher-degree polynomials?
    \item Repeat the steps above, but this time use the \emph{numpy.polyfit()} method to fit a model to the data and compare it to the previous models
\end{itemize}

%\bibliographystyle{alpha}
%\bibliography{sample}

\end{document}
