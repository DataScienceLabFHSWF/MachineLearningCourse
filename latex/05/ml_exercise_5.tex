\documentclass{article}

% Language setting
% Replace `english' with e.g. `spanish' to change the document language
\usepackage[english]{babel}

% Set page size and margins
% Replace `letterpaper' with`a4paper' for UK/EU standard size
\usepackage[letterpaper,top=2cm,bottom=2cm,left=3cm,right=3cm,marginparwidth=1.75cm]{geometry}

% Useful packages
\usepackage{amsmath}
\usepackage{graphicx}
\usepackage[colorlinks=true, allcolors=blue]{hyperref}

\title{Machine Learning \\ Exercise 5: Dimensionality Reduction with PCA}
\author{Prof. Dr. Thomas Kopinski}

\begin{document}
\maketitle

\begin{abstract}
In this exercise you will learn how to use \emph{Principal Component Analysis} (PCA) to extract meaningful information from different datasets by reducing the dimension of the provided data.
\end{abstract}

\section*{Task 1:}
\begin{itemize}
    \item Please download the Jupyter notebook for this exercise from \href{https://github.com/DataScienceLabFHSWF/MachineLearningCourse/tree/main/exercises/05}{here} as it contains useful information, code snippets, help and some directions for the following tasks. Additionaly please download the "wine.data" dataset from \href{https://github.com/DataScienceLabFHSWF/MachineLearningCourse/tree/main/data/05}{here}. 
    \item Work through the examples in the notebook up to the section \emph{Task 2}. 
    \item Additional information about the \emph{PCA} can be found in the course material.
\end{itemize}

\section*{Task 2: Iris dataset}
\begin{itemize}
    \item In this task you will implement various classifiers to predict the species of Iris flowers, on the original, preprocessed data as well as on the PCA-transformed ones.
    \item Download the "IRIS.csv" dataset from \href{https://github.com/DataScienceLabFHSWF/MachineLearningCourse/tree/main/data/05}{here}.
    \item Load the dataset into a dataframe and get an overview about its content.
    \item Visualize different aspects of the dataset (e.g. class distribution, distribution of the individual features)
    \item Use the seaborn package to plot a correlation matrix of the dataframe (seaborn.heatmap())
    \item Preprocess the data.
    \item Choose several suitable classifiers, train the models and compare and visualize the results for untransformed and transformed data.
\end{itemize}

\section*{Task 3: Credit dataset}
\begin{itemize}
    \item This time you have to deal with the "credit" dataset which can be found \href{https://github.com/DataScienceLabFHSWF/MachineLearningCourse/tree/main/data/05}{here}. 
    \item Load the dataset into a dataframe and take a look at it. Which are the feature columns, which column the target? Any correlations?
    \item Try to deal with the missing data (replace, delete, etc.)
    \item Encode the variable features.
    \item Use PCA on the cleaned dataframe. What are your findings? How many components would you choose?
    \item Train, test, evaluate and plot various combinations of models and PCA-transformations.
\end{itemize}

%\bibliographystyle{alpha}
%\bibliography{sample}

\end{document}
