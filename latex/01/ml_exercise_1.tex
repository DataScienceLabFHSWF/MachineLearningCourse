\documentclass{article}

% Language setting
% Replace `english' with e.g. `spanish' to change the document language
\usepackage[english]{babel}

% Set page size and margins
% Replace `letterpaper' with`a4paper' for UK/EU standard size
\usepackage[letterpaper,top=2cm,bottom=2cm,left=3cm,right=3cm,marginparwidth=1.75cm]{geometry}

% Useful packages
\usepackage{amsmath}
\usepackage{graphicx}
\usepackage[colorlinks=true, allcolors=blue]{hyperref}

\title{Machine Learning \\ Exercise 1: Introduction to linear regression with scikit-learn and scipy}
\author{Prof. Dr. Thomas Kopinski}

\begin{document}
\maketitle

\begin{abstract}
In this week's exercise you will become familiar with using the python module scikit-learn to perform a linear regression on a given dataset. In addition you will use the more flexible curve\_fit function from the scipy package. The data can be checked out from this courses's Git repository. 
\end{abstract}

\section*{Task 1: Getting familiar with scikit-learn}

\begin{itemize}
    \item Additional information about implementing linear regression in python with the help of scikit-learn can be found  \href{https://github.com/DataScienceLabFHSWF/machine-learning-book/blob/main/notebooks/ch09/ch09.ipynb}{here}
    \item The repository for this course can be found \href{https://github.com/DataScienceLabFHSWF/MachineLearningCourse}{here}
\end{itemize}
\section*{Task 2: Linear regression with scikit-learn}
\begin{itemize}
    \item Import the necessary modules
    \item Load the file with the name "simple\_regression.csv" from the data subfolder with numpy or pandas 
    \item Instantiate a LinearRegression model from the scikit-learn library
    \item Fit a linear model $f(x) = y = mx+b$ to the given data (reshaping of the data might be necessary)
    \item Display the values of the coefficients and plot the data and fitted function
    \item Interpret the results
\end{itemize}

\section*{Task 3: Linear regression with scipy}
\begin{itemize}
    \item Import the necessary modules
    \item Load the file with the name "simple\_regression.csv" from the data subfolder with numpy or pandas 
    \item Define a model function that returns a linear mapping of the variable $x$
    \item Fit that linear model to the data with scipy.optimize.curve\_fit
    \item Display the values of the coefficients and covariance matrix
    \item What is the meaning of the covariance matrices entries?
    \item Plot the data and fitted function
    \item Interpret the results and compare it to the scikit-learn version
    \item Can this method be used in a more general way?
\end{itemize}

%\bibliographystyle{alpha}
%\bibliography{sample}

\end{document}
